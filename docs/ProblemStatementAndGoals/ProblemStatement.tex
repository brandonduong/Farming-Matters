\documentclass{article}

\usepackage{tabularx}
\usepackage{booktabs}

\title{Problem Statement and Goals\\\progname}

\author{\authname}

\date{}

%% Comments

\usepackage{color}

\newif\ifcomments\commentstrue %displays comments
%\newif\ifcomments\commentsfalse %so that comments do not display

\ifcomments
\newcommand{\authornote}[3]{\textcolor{#1}{[#3 ---#2]}}
\newcommand{\todo}[1]{\textcolor{red}{[TODO: #1]}}
\else
\newcommand{\authornote}[3]{}
\newcommand{\todo}[1]{}
\fi

\newcommand{\wss}[1]{\authornote{blue}{SS}{#1}} 
\newcommand{\plt}[1]{\authornote{magenta}{TPLT}{#1}} %For explanation of the template
\newcommand{\an}[1]{\authornote{cyan}{Author}{#1}}

%% Common Parts

\newcommand{\progname}{ProgName} % PUT YOUR PROGRAM NAME HERE
\newcommand{\authname}{Team \#, Team Name
\\ Student 1 name
\\ Student 2 name
\\ Student 3 name
\\ Student 4 name} % AUTHOR NAMES                  

\usepackage{hyperref}
    \hypersetup{colorlinks=true, linkcolor=blue, citecolor=blue, filecolor=blue,
                urlcolor=blue, unicode=false}
    \urlstyle{same}
                                


\begin{document}

\maketitle

\begin{table}[hp]
\caption{Revision History} \label{TblRevisionHistory}
\begin{tabularx}{\textwidth}{llX}
\toprule
\textbf{Date} & \textbf{Developer(s)} & \textbf{Change}\\
\midrule
 09/25/2022 & Namit Chopra, Brandon Duong  & Finished first \\
 & Andrew Balmakund, Mohammad Harun & version\\
 & Mihail Serafimovski & \\
 04/01/2023 & Namit Chopra, Brandon Duong  & Final reviewed \\
 & Andrew Balmakund, Mohammad Harun & version \\
 & Mihail Serafimovski & \\
\bottomrule
\end{tabularx}
\end{table}

\section{Problem Statement}


\subsection{Problem}
Conducting survey research through traditional methods has some setbacks such as respondents' biases and their reluctance to provide an answer, as well as the overall 'artificial' feeling of a survey. The questions are also too broad and generalized and don't necessarily provide insight into real-life decisions. Users often prioritize finishing the survey or experiment as quickly as possible instead of providing genuine answers. The lack of motivation to do surveys or lab experiments skews the results.

The solution will mask traditional conventions of conducting surveys and lab experiments. It will be interactive and engaging, allowing users to make genuine decisions that can be used for research.

\subsection{Inputs and Outputs}
The inputs will be all user interactions in the app. The outputs will be data representing the user's behaviour regarding risk and reward. The output data is obtained from all of the fundamental choices the user makes that are required for research. 

\subsection{Stakeholders}
    \begin{itemize}
        \item \textbf{Dr.Yiannakoulias}: He is a direct stakeholder in this project. His experiment on understanding human decision-making is the main reason for the idea behind this project. He can be thought of as a client in this scenario.
        \item \textbf{Anyone interested in understanding how humans make decisions}: These would be indirect stakeholders as they might be interested in the results of the experiment.
        \item \textbf{Developers}: As the developers of the game, we would be considered indirect stakeholders. We are not as much concerned about the experiment itself but rather the game functionality and performance. 
        \item \textbf{Users who will play the game}: These would be direct stakeholders as they will be a part of the experiment and the results of their decisions will be collected and analyzed by Dr.Yiannakoulias.
    \end{itemize}

\subsection{Environment}
The application must run on smart phones, laptops, tablets, and desktops running the latest version of their respective OS (Windows, MacOS, Linux). The game shall be able to be played on any computer with access to the internet. It will be supported on all modern browsers (at least Chrome, Edge, Firefox).

\section{Goals}
    \begin{itemize}
        \item \textbf{Secure Personal Information}: All personal information provided by the users will be kept secured in a database. This is important because the users will complete a consent form which acknowledges their privacy. As well, this app should support the removal of user data. 
        \item \textbf{In-game shop}: Players will be able to buy items to plant and build up their farm, and sell items and crop for money. This will be important as the game must mask the experiment, of which aims to help researchers gain unbiased data to better understand human decisions when it comes to risk-taking.
        \item \textbf{Customizable Farm}: Players will be able to customize their farm buildings and crops. By giving players more freedom, the experience is more enjoyable and players are further immersed in the game. With more decisions, the player is less likely to know which decisions are the ones pertaining to the experiment and therefore provide realistic results.
        \item \textbf{Provide a login system for users}: This will ensure that users can pick up where they left off and continue playing the game at a later time. This will also allow all data to be traced back to the player, in the event that a user requests their data to be erased.
        \item \textbf{Log and store critical player decisions}: This is a crucial part of the project as certain decisions logged on the server will be used for the research. The decisions will also need to be saved and stored once the user stops playing the game. This will ensure the logged player decisions can be analyzed and reviewed at any time.
        \item \textbf{Make an engaging and interactive game}: This is very significant as users who want to play the game will provide more meaningful data. User surveys and in-game metrics will be used to get an idea of how motivated users are to play the game (i.e number of turns played, time a player took making their decisions during a turn).
        \item \textbf{Provide a summary of events}: This will give users a recap of what they did each turn (what did they plant, buy, sell, build, etc.) and may help them with their decisions in each consecutive turn. 
        \item \textbf{Game Audio}: The game will have background music, and sound effects which will give users feedback after each action. This will help create a more immersive experience.
    \end{itemize}
\section{Stretch Goals}
    \begin{itemize}
        \item \textbf{Asynchronous multiplayer}: This will be an exciting feature as players will be able to compete with other players in terms of how efficient they were with their turns, and what decisions the other players made. Multiplayer gameplay is usually a feature found in most games today; however, it will be asynchronous so players don't have to wait on others before they can perform actions.
        \item \textbf{Mini-games}: This will be a fun in-game feature which aims to keep players enticed. Mini-games could be played for various reasons. This could be an enjoyable option to earn money for buying items required for their farm. It is also an additional feature to keep the users engaged and motivated to continue playing the game.
        \item \textbf{Visualizing the data collected by the game}: The data logged and sent to databases could be graphed and visualized in many other ways. This would also be useful for analyzing the results for the experiment.
    \end{itemize}
\end{document}