\documentclass{article}

\usepackage{tabularx}
\usepackage{booktabs}

\title{Problem Statement and Goals\\\progname}

\author{\authname}

\date{}

\input{../Comments}
%% Common Parts

\newcommand{\progname}{Farming Matters} % PUT YOUR PROGRAM NAME HERE
\newcommand{\authname}{Team \#14, The Farmers
\\ Brandon Duong
\\ Andrew Balmakund
\\ Mihail Serafimovski
\\ Mohammad Harun
\\ Namit Chopra} % AUTHOR NAMES                  

\usepackage{hyperref}
    \hypersetup{colorlinks=true, linkcolor=blue, citecolor=blue, filecolor=blue,
                urlcolor=blue, unicode=false}
    \urlstyle{same}
                                


\begin{document}

\maketitle

\begin{table}[hp]
\caption{Revision History} \label{TblRevisionHistory}
\begin{tabularx}{\textwidth}{llX}
\toprule
\textbf{Date} & \textbf{Developer(s)} & \textbf{Change}\\
\midrule
Date1 & Name(s) & Description of changes\\
Date2 & Name(s) & Description of changes\\
... & ... & ...\\
\bottomrule
\end{tabularx}
\end{table}

\section{Problem Statement}

\subsection{Problem}
Conducting survey research through traditional methods has some setbacks such as respondents' biases and their reluctance to provide an answer, as well as the overall 'artificial' feeling of a survey. The questions are also too broad and generalized and don't necessarily provide insight into real-life decisions. Users often prioritize finishing the survey or experiment as quickly as possible instead of providing genuine answers. The lack of motivation to do surveys or lab experiments skews the results.

The solution will mask traditional conventions of conducting surveys and lab experiments. It will be interactive and engaging allowing users to make genuine decisions that can be used for research.

\subsection{Inputs and Outputs}
The inputs will be all user interactions in the app. The outputs will be data representing the user's behaviour regarding risk and reward. The output data is obtained from all of the fundamental choices the user makes that are required for research. 

\subsection{Stakeholders}

\subsection{Environment}

\wss{Hardware and software}

\section{Goals}
\begin{itemize}
    \item \textbf{Secure Personal Information}: All personal information provided by the users will be kept secured in a database. This is important because the users will complete a consent form which acknowledges their privacy. As well, this app should support the removal of user data. 
    \item \textbf{In-game shop}: Players will be able to buy and sell items to customize their farm. This will be important as the experiment aims to better understand human decisions when it comes to risk-taking.
    \item \textbf{Customizable Farm}: By giving players more freedom, it does not restrict their gameplay and aims to make the experience more enjoyable. Also, the experiment aims to add variability to make the results more realistic.
    \item \textbf{Provide a login system for users}: This will ensure that users can pick off where they left off and continue playing the game at a later time. 
    \item \textbf{Log and storing critical player decisions}: This is a crucial part of the project as certain decisions logged on the server will be used for the research. The decisions will also need to be saved and stored once the user stops playing the game. This will ensure the logged player decisions can be analyzed and reviewed at any time.
    \item \textbf{Make an engaging and interactive game}: This is very significant as users who want to play the game will provide more meaningful data. User surveys and in-game metrics will be used to get an idea of how motivated users are to play the game.
    \item \textbf{Provide a summary of events}: This will give users a recap of each turn and may help them with their decisions in each consecutive turn. 
    \item \textbf{Game Audio}: The game will have a soundtrack which will give users feedback after each action. This will help create a more immersive experience.
\end{itemize}
\section{Stretch Goals}
\begin{itemize}
    \item \textbf{Asynchronous multiplayer}: This will be an exciting feature as players will be able to collaborate with other players. Multiplayer gameplay is usually a feature found in most games today; however, it will be asynchronous so players don't have to wait on others before they can perform actions.
    \item \textbf{Mini-games}: This will be a fun in-game feature which aims to keep players enticed. Mini-games could be played for various reasons. This could be an enjoyable option to earn money for buying items required for their farm. It is also an additional feature to keep the users engaged and motivated to continue playing the game.
    \item \textbf{Visualizing the data collected by the game}: The data logged and sent to databases could be graphed and visualized in many other ways. This would also be useful for collecting results for the experiment.
\end{itemize}
\end{document}
