\documentclass[12pt]{article}

\usepackage{hyperref}
\hypersetup{colorlinks=true,
    linkcolor=blue,
    citecolor=blue,
    filecolor=blue,
    urlcolor=blue,
    unicode=false}
\urlstyle{same}

\usepackage{enumitem,amssymb}
\newlist{todolist}{itemize}{2}
\setlist[todolist]{label=$\square$}
\usepackage{pifont}
\newcommand{\cmark}{\ding{51}}%
\newcommand{\xmark}{\ding{55}}%
\newcommand{\done}{\rlap{$\square$}{\raisebox{2pt}{\large\hspace{1pt}\cmark}}%
\hspace{-2.5pt}}
\newcommand{\wontfix}{\rlap{$\square$}{\large\hspace{1pt}\xmark}}

\begin{document}

\title{MIS Checklist}
\author{Spencer Smith}
\date{\today}

\maketitle

% Show an item is done by   \item[\done] Frame the problem
% Show an item will not be fixed by   \item[\wontfix] profit

\begin{itemize}
 
\item Follows writing checklist (full checklist provided in a separate document)
  \begin{todolist}
  \item \LaTeX{} points
  \item Structure
  \item Spelling, grammar, attention to detail
  \item Avoid low information content phrases
  \item Writing style
  \end{todolist}

\item MIS Module Classifications
  \begin{todolist}
  \item Types that only hold data (records) are modelled as exported types.  For
    instance, the StdntAllocTypes module in A2:
    \url{https://gitlab.cas.mcmaster.ca/smiths/se2aa4_cs2me3/blob/master/Assignments/A2/A2.pdf})
  \item Types that have data (state) and behaviour are modelled as ADTs.  The
    MIS should use the keyword \textbf{Template}.  An example is the BoardT ADT
    given at
    \url{https://gitlab.cas.mcmaster.ca/smiths/se2aa4_cs2me3/blob/master/Assignments/A3/A3Soln/A3P1_Spec.pdf}
  \item Abstract objects are used when there is only one instance.  There is
    state and behaviour.  This most
    often comes up for ``global'' reader and writer modules.  For instance, a
    module that does logging.  Abstract objects do NOT use the word Template in
    the main header.  An example is given in the SALst module of A2:
    \url{https://gitlab.cas.mcmaster.ca/smiths/se2aa4_cs2me3/blob/master/Assignments/A2/A2.pdf}
    \item Library modules are used when there is only behaviour, no state.  They
      are defined as Modules, but State Variables and Environment Variable
      fields say ``None.''
  \item If the module's MIS can be parameterized by type, then the keyword
    \textbf{Generic} is used.  Generic modules are usually also Template
    modules, but not necessarily.  An example is given in the Generic Stack
    Module (Stack(T)) given in A3:
    \url{https://gitlab.cas.mcmaster.ca/smiths/se2aa4_cs2me3/blob/master/Assignments/A3/A3Soln/A3P1_Spec.pdf}
  \item Abstract objects will have some kind of initialization method
    \item Abstract objects will have an assumptions that programmers will
      initialize first, or a state variable that is set from False to True when
      the Abstract object is initialized - this state variable then needs to be
      checked for each access program call
  \end{todolist}

  \item MIS and Mathematical syntax
  \begin{todolist}
  \item Exported constants are ``hard-coded'' literal values, not variables.
    Constants are values that are known at specification (and therefore compile)
    time.  Explicit constant values are provided in the MIS, not left to be
    filled in later.  (They can be changed later, but specific values should be given.)
  \item Operations do not mix incorrect types.  For instance, a character is not
    added to an integer, an integer is not ``anded'' with a Boolean, etc.
  \item Our
    \href{https://gitlab.cas.mcmaster.ca/smiths/se2aa4_cs2me3/-/blob/master/MISFormat/MISFormat.pdf}
    {modified Hoffmann and Strooper notation} is used, or any new notation is
    clearly defined.
  \end{todolist}

\item MIS Semantics for each module
  \begin{todolist}
  \item Each access program does something - either an output, or a state
    transition
  \item Access programs either change the state of something, or have an output.
    Only rarely should an access program do both (as it does for the constructor
    in an ADT.)
  \item If there is an entry in the state transition, then the state of
    something changes.  The state change might be the local state variables, the
    state variables for another module, or an environment variable.
  \item Outputs use $out := ...$
  \item Exceptions use $exc := ...$
  \item If the state invariant is satisfied before an access program call, it
    will remain satisfied after the call
  \item State invariant is initially satisfied
  \item Local functions make the specification easier to read (there is no
    requirement that the local functions will actually be implemented in code)
  \item Modules that deal with files, the keyboard, or the screen, have
    environment variables to represent these respective entitites
  \item Symbols are from SRS - not yet translated to code names (that is use
    $\theta$, not \texttt{theta})
  \end{todolist}

\item MIS Quality inspection for each module
  \begin{todolist}
  \item Consistent
  \item Essential
  \item General
  \item Implementation independent
  \item Minimal
  \item High cohesion
  \item Opaque (information hiding)
  \end{todolist}

\item MIS Completeness
  \begin{todolist}
  \item All types introduced in the spec are defined somewhere
  \item All modules in MG are in the MIS
  \item All required sections of the template are present for all modules
  \end{todolist}

\end{itemize}

\end{document}
