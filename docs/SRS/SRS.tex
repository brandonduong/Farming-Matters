\documentclass{article}

\usepackage{booktabs}
\usepackage{tabularx}
\usepackage{enumerate}
\usepackage{graphicx}
\usepackage{float}
\usepackage{chngpage}
\usepackage[round]{natbib}

\title{Software Requirements Specification\\\progname}

\author{\authname}

\date{}

\input{../Comments}
%% Common Parts

\newcommand{\progname}{Farming Matters} % PUT YOUR PROGRAM NAME HERE
\newcommand{\authname}{Team \#14, The Farmers
\\ Brandon Duong
\\ Andrew Balmakund
\\ Mihail Serafimovski
\\ Mohammad Harun
\\ Namit Chopra} % AUTHOR NAMES                  

\usepackage{hyperref}
    \hypersetup{colorlinks=true, linkcolor=blue, citecolor=blue, filecolor=blue,
                urlcolor=blue, unicode=false}
    \urlstyle{same}
                                


\begin{document}

\begin{table}[hp]
\caption{Revision History} \label{TblRevisionHistory}
\begin{tabularx}{\textwidth}{llX}
\toprule
\textbf{Date} & \textbf{Developer(s)} & \textbf{Change}\\
\midrule
09/28/2022 & Namit Chopra, Brandon Duong  & Finished First\\
 & Andrew Balmakund, Mohammad Harun &  Version\\
 & Mihail Serafimovski & \\
\bottomrule
\end{tabularx}
\end{table}

\newpage

\maketitle

\newpage

\pagenumbering{roman}
\tableofcontents
\listoftables
\listoffigures
\newpage

\pagenumbering{arabic}

This document describes the requirements for Farming Matters. The template for the Software
Requirements Specification (SRS) is a subset of the Volere ~\citep{RobertsonAndRobertson2012} template. \newline \\ Some modifications made to the Volere subset template are:
\begin{itemize}
    \item (Added) Section 3.10: Requirements that are likely/unlikely to change
    \item (Added) Section 4: Traceability matrix
    \item (Added) Section 5.5: Requirements Phase-In Plan
    \item (Added) Section 6.1: Team technologies/skills - Reflection
    \item (Removed) Section 5.11 Ideas for Solutions - Ideas for solutions are represented in the risks section, as mitigation is provided for each risk.
\end{itemize}

\section{Project Drivers}

\subsection{The Purpose of the Project}
The purpose of this project is to conduct survey research through an interactive and engaging activity. 
This will further help understand genuine decisions from the users to help with the research of understanding risk-making decisions.

\subsection{The Stakeholders}

\subsubsection{The Client}
The client of this project is Dr.Yiannakoulias who is the supervisor of this project. 
Dr.Yiannakoulias is part of the School of Earth, Environment and Society McMaster University.

\subsubsection{The Customers}
The customers of the project are individuals that enjoy management and role-playing simulation games. 
As well as individuals willing to be a respondent to help conduct data for research.

\subsubsection{Other Stakeholders}
Other stakeholders would include the Ethics board. 
Some of the requirements regarding data collection are derived from this stakeholder. 
They also have to approve the final product to check if there are any ethical issues.

\subsection{Mandated Constraints}

\subsection{Naming Conventions and Terminology}
\begin{itemize}
  \item Player: The user playing the game. The player or user is the participant and focus of the study.
  \item Land: An area where the user can interact with the farm. This includes planting crops, fertilizing crops and adding buildings.
  \item Inventory: Where the user will be able to store items.
  \item Items: The user will be able to acquire these into their inventory, including grown crops, crop seeds, and fertilizers.
  \item Focus groups: A set of users will be involved with discrete decision-making and another set of users will be involved with probabilistic decision-making.
  \item Turns: these are rounds that happen per season where a decision can be made. 
  \item Seasons: Including Winter, Spring, Summer, and Fall. The current season changes depending on the turn number, and has an effect on which crops can be grown. Each season lasts SEASON\_LENGTH turns.
  \item Key Questions: These are the compulsory questions that will be asked to the user. The first will be whether the user wants to pay the consultant for advice. The second question will ask the user if they want to purchase insurance for crops.
\end{itemize}

\subsection{Relevant Facts and Assumptions}

The user should have the physical and visual ability to operate a computer. 
This includes actions such as clicking a mouse and pressing buttons on the keyboard. 
The user should be familiar with the hardware they are using, in this case, a desktop or laptop. 
It would be preferable if the user had some knowledge of the basic idea of farming; however, there 
will be a tutorial to go over the basic rules of the game if they lack sufficient knowledge on how to play. 
\\ Due to the research-oriented nature of the project and the ethical requirements, all users will be over the age of 18.

\section{Functional Requirements}

\subsection{The Scope of the Work and the Product}

\subsubsection{The Context of the Work}


\subsubsection{Work Partitioning}
\begin{table}[h]
    \centering
    \begin{tabular}{|p{0.33\linewidth} | p{0.33\linewidth} | p{0.33\linewidth}| }
    \hline
         Event & Input/Output & Summary \\
         \hline
         User ends turn & Input: user selects end turn option & System responds and update current game state\\
         \hline 
         Collect user decisions & Input: all prior user decisions, users progression and focus group \newline Output: Group data based on focus group  & System provides organized data \\
         \hline 
          User creates account & Input: Username and password \newline Output: Associated account & The user inputs their desired username and password and the system creates an account and stores the credentials\\
         \hline 
         User requests to delete their data & Input: Username and password \newline Output: Associated account and all its data is deleted & The user inputs their account's username and password and the system deletes the associated account and all its corresponding data\\
         \hline 
    \end{tabular}
    \caption{Work Partitioning Table}
    \label{tab:my_label}
\end{table}
\newpage

\subsubsection{Individual Product Use Cases and Undesired Event Handling}

\textbf{Use case \#1:} User creates account\\
\textbf{Primary Actor:} User\\
\textbf{Supporting Actors:} None\\
\textbf{Precondition:} The user has completed the consent form\\
\textbf{Trigger:} The user is directed to the create account page\\
\textbf{Main Success Scenario}
\begin{enumerate}
    \item User is on the home page and selects to create account
    \item User is on the create account page
    \item User provides the required information
    \item System verifies all required information has been provided
    \item User completes human verification task
    \item System validates human verification
    \item System securely registers user information
    \item User is redirected back to the Home page
\end{enumerate}
\textbf{Secondary Scenarios}:
\begin{enumerate}
    \item User fails to provide the required information: one or more fields have not been provided by the user
    \item User fails human verification task: user has failed a human verification test which either means they entered in the wrong details or an automated script attack for account creation occurred

\end{enumerate}
\textbf{Success Postcondition:} The user has successfully created an account and account information is stored and secured in a database.


\noindent\\
\textbf{Use case \#2:} User signs into account\\
\textbf{Primary Actor:} User\\
\textbf{Supporting Actors:} None\\
\textbf{Precondition:} The user has successfully created an account\\
\textbf{Trigger:} The user arrives at the login page\\
\textbf{Main Success Scenario}
\begin{enumerate}
    \item User is on the home page and selects to login
    \item User is on the login page
    \item User provides the required information
    \item System verifies all required information has been provided
    \item User is redirected to the Home page as a logged in user
\end{enumerate}
\textbf{Secondary Scenarios}:
\begin{enumerate}
    \item Login unsuccessful: the provided login details do not match any login details in the database
\end{enumerate}
\textbf{Success Postcondition:} The user has successfully logged into their created account and their game state is loaded from the database. \newline

\noindent\\
\textbf{Use case \#3:} User logs out of their account\\
\textbf{Primary Actor:} User\\
\textbf{Supporting Actors:} None\\
\textbf{Precondition:} The user has successfully created an account and is currently logged in\\
\textbf{Trigger:} The user has selected an option to logout\\
\textbf{Main Success Scenario}
\begin{enumerate}
    \item User accesses and selects the logout option
    \item User is presented with a confirmation page to logout
    \item User selects logout option
    \item System shall save all progress made in the current session and append to their overall progression that is stored in the database
    \item User is redirected to Home page
\end{enumerate}
\textbf{Secondary Scenarios}:
\begin{enumerate}
    \item User is inactive: After INACTIVE\_TIME period, the system shall log out the user automatically. Perhaps they may have closed the tab running the web application and have not returned for a while
\end{enumerate}
\textbf{Success Postcondition:} The user has successfully logged out of the system and all logged data has been stored successfully. \newline


\noindent\\
\textbf{Use case \#4:} User retrieves forgotten password\\
\textbf{Primary Actor:} User\\
\textbf{Supporting Actors:} None\\
\textbf{Precondition:} The user has successfully created an account\\
\textbf{Trigger:} The user has selected an option to forget password\\
\textbf{Main Success Scenario}
\begin{enumerate}
    \item User accesses login page and selection forget password option
    \item User enters required information to gain back access to account
    \item System sends the users follow up details to confirm the legitimacy of their account
    \item User is able to provide a new password
    \item System shall update the password of the current user in the database 
    \item User is redirected to the Home page as a logged in user
\end{enumerate}
\textbf{Secondary Scenarios}:
\begin{enumerate}
    \item System unable to verify account details: User may have entered the incorrect details needed to obtain back access to the account. They may also pretend to be someone else but is not able to gain access to the account
\end{enumerate}
\textbf{Success Postcondition:} The user has successfully updated their account details and is logged in. \newline

\noindent\\
\textbf{Use case \#5:} User opts out to delete their data\\
\textbf{Primary Actor:} User\\
\textbf{Supporting Actors:} None\\
\textbf{Precondition:} The user has successfully created an account and is currently signed in\\
\textbf{Trigger:} The user has selected an option to delete their data\\
\textbf{Main Success Scenario}
\begin{enumerate}
    \item User accesses selection to delete data
    \item System logs the user out
    \item System deleted all data belonging to the user
    \item User is redirected to the Home page
\end{enumerate}
\textbf{Secondary Scenarios}:
None\\
\textbf{Success Postcondition:} The user has successfully opted out to delete their data and is no longer stored in the database. \newline


\noindent\\
\newpage
\textbf{Use case\# 6:} User completes a turn\\
\begin{figure}[H]
    \centering
    \includegraphics[width=12cm]{srs_usecase.drawio.png}
    \caption{Use Case \#6: User completes a turn}
    \label{fig:context_diagram}
\end{figure}
\newpage

\subsection{Functional Requirements}
\begin{enumerate}[{FR}1. ]
  %Other
  \item The system must allow users to create an account.\\
  \textbf{Rationale}: The purpose of the game is to track user decisions and be able to see which participant did what decision. For this to happen, users must be associated to an account.
  \item The system must allow the user to reset their password.\\
  \textbf{Rationale}: It is possible a user misplaces their password and loses all their account progress. This effectively removes the participant from the study and so this must be rectified.
  \item The system must allow users to accumulate in-game currency.\\
  \textbf{Rationale}: In-game currency acts as a performance metric, and also helps engage the player into planning ahead for their future decisions. Optimizing what one should spend their in-game currency on is both part of the study, and the engagement of the game itself.
  \item The system must allow the user to store their items in an inventory.
  \textbf{Rationale}: The user must be able to view the items they currently own.
  \item The system must verify new users as human.\\
  \textbf{Rationale}: As anyone can make an account, the system is susceptible to related attacks. The system verifying a user as human before creating their account may slow down these attempts.
  %Shop related
  \item The system must allow users to purchase items from a shop.\\
  \textbf{Rationale}: Helps engage the player with the feeling of progression and decision-making.
  %\item The system must allow users to buy crop seeds\\
  %\textbf{Rationale}: Helps engage the player with the feeling of progression and decision-making
  \item The system must allow users to grow crop on owned land.\\
  \textbf{Rationale}: Helps engage the player with the feeling of progression and decision-making.
  \item The system must allow users to sell crop at a fluctuating price.\\
  \textbf{Rationale}: Helps engage the player with the feeling of progression and decision-making.
  %\item The system must allow users to buy fertilizer\\
  %\textbf{Rationale}: Helps engage the player with the feeling of progression and decision-making
  \item The system must allow users to use fertilizer on planted crop.\\
  \textbf{Rationale}: Helps engage the player with the feel of progression and decision-making.
  %Playing the game
  \item The system must allow users to buy land.\\
  \textbf{Rationale}: Helps engage the player with the feel of progression and decision-making.
  \item The system must prompt users for consulting advice every CONSULTING\_INTERVAL turns.\\
  \textbf{Rationale}: This decision is one of the two key questions the research is looking at. Whether people are willing to pay for information, and for what information (i.e deterministic or probabilistic).
  \item The system must prompt users for insurance for planted crops.\\
  \textbf{Rationale}: This decision is the second of the two key questions the research is looking at. Whether people are willing to pay for insurance, and in what circumstance (i.e from the start, or only after they've been affected by an insurable event).
  \item The system must be able to log user decisions.\\
  \textbf{Rationale}: This was a crucial requirement that was provided by Dr.Yiannakoulias who will be able to view these logs to analyze the data.
  \item The system must be able to save user game state.\\
  \textbf{Rationale}: This will give users the options to continue where they left off if they want to continue playing at another time.
  \item \label{FR15} The system must allow users to delete their data.\\
  \textbf{Rationale}: This is a requirement from the ethics board and it is mandatory that users be given this option.
  \item The system must have a defined area of land for users to manage farm.\\
  \textbf{Rationale}: This will provide an environment in which players can visualize and interact with their farm.
  \item The system must allow users to place buildings and items on their land.\\
  \textbf{Rationale}: This will give the user more variability and options in terms of how they want to build their farm.
  \item \label{FR18} The system must present the consent form to participate in the study before starting the game.\\
  \textbf{Rationale}: This is another requirement from both the Ethics board and Dr.Yiannakoulias, and made this a point to emphasize.
  \item The system must be able to evaluate the total worth of a user's assets.
  \textbf{Rationale}: Total worth of all a user's assets acts as a performance metric, and also helps engage the player by allowing them to see how much they've progressed.
  \item The system must be able assign a user to specific focus group.\\
  \textbf{Rationale}: This is essential for the research study as to demonstrate whether or not people prefer deterministic or probabilistic information.
  \item The system must include random events to occur EVENT\_OCCURRENCE.\\
  \textbf{Rationale}: Helps engage the player by needing them to plan ahead for the possible unknown.
  \item The system must have the current season change over time.\\
  \textbf{Rationale}: Helps engage the player as they must plan ahead for what is possible during the different seasons in terms of what to plant, and what risks there are within each season.
\end{enumerate}

\subsection{Requirements That Are Likely/Unlikely to Change}
\begin{table}[H]
  \centering
  \begin{tabular}{|p{0.40\linewidth} | p{0.40\linewidth}|}
  \hline
       Likely & Unlikely \\
       \hline
       FR2, FR5, FR19, FR21 & FR1, FR3, FR4, FR6, FR7, FR8, FR9, FR10, FR11, FR12, FR13, FR14, FR15, FR16, FR17, FR18, FR20, FR22    \\
       \hline
  \end{tabular}
  \caption{Likely/Unlikely to Change Table}
  \label{tab:my_label}
\end{table}
\begin{itemize}
  \item FR1: This is unlikely to change as accounts are needed to associate a specific user to their game state
  \item FR3, FR4, FR6, FR7, FR8, FR9, FR10, F14, FR16, F17, FR21, FR22: These are unlikely to change as these are the basic core game mechanics that define the desired game loop
  \item FR11, FR12, FR13, FR20: These are unlikely to change as they are implement the essentials behind the research study aspect
  \item FR15, F18: This is unlikely to change because it is required for the ethics board to approve the game. The approval of the game by the ethics board is a must.
  
\end{itemize}
\section{Non-functional Requirements}

\subsection{Look and Feel Requirements}
\begin{enumerate}[{LF}1. ]
    \item The menu shall be minimalistic. \\
    \textbf{Fit Criteria}: The menu should only require the necessary elements and not overwhelm the user.
    \item The menu and game interface shall follow a consistent theme. \\
    \textbf{Fit Criteria}: All visual assets should include the same colors and style across all aspects of the game.  
    \item The system must have engaging audio. \\
    \textbf{Fit Criteria}: Survey a group of individuals and 90\% of them should be satisfied with the audio assets. 
    \item The system must have engaging graphics.\\
    \textbf{Fit Criteria}: Survey a group of individuals and 90\% of them should be satisfied with the graphical assets.
    \item The system must be responsive on monitors at least MIN\_MONITOR. \\
    \textbf{Fit Criteria}: All aspects of the system should be functioning accordingly and be responding at the resolution set to MIN\_MONITOR.
    \item The system must have engaging gameplay. \\
    \textbf{Fit Criteria}: Average number of turns played is MIN\_TURNS.
    
\end{enumerate}

\subsection{Usability and Humanity Requirements}
\begin{enumerate}[{UH}1. ]
    \item The system must be easy to learn for people aged AGE\_GROUP. \\
    \textbf{Fit Criteria}: Survey a group of individuals and 95\% of them should easily understand the in-game logic, understand the information provided to them and be able to interact with all the menus.
\end{enumerate}

\subsection{Performance Requirements}
\begin{enumerate}[{PR}1. ]
    \item The website must load within LOAD\_TIME seconds. \\
    \textbf{Fit Criteria}: The measured time from the user sending an initial request to access the website and the website loading is less than or equal to LOAD\_TIME.
    
    \item The website must respond to any user interaction within RESPONSE\_TIME seconds. \\
    \textbf{Fit Criteria}: The measured time from the user interacting with the website and the website providing the user with some response is less than or equal to RESPONSE\_TIME.
\end{enumerate}

\subsection{Operational and Environmental Requirements}
\begin{enumerate}[{OE}1. ]
    \item The system must be functional on modern browsers. \\
    \textbf{Fit Criteria}: The user shall be able to run the game on different modern browsers such as Google Chrome, Firefox, Microsoft Edge.
    
    \item The system must be functional on any version of the browser it's used on released within the past SUPPORTED\_VERSIONS. \\
    \textbf{Fit Criteria}: The user shall be able to run the game on any supported version of their browser of choice (so long as their browser is supported).
\end{enumerate}


\subsection{Maintainability and Support Requirements}
\begin{enumerate}[{MR}1. ]
    \label{MR1}
    \item The system must allow assets to be replaced. \\
    \textbf{Fit Criteria}: The development team and client shall be able to change any visual or audio assets with ease and not disrupt any part of the system.
\end{enumerate}

\subsection{Security Requirements}
\begin{enumerate}[{SR}1. ]
    \item The system must not allow automation of creating accounts. \\
    \textbf{Fit Criteria}: There should be measures in place to prevent automated attacks. \\
    
\end{enumerate}
\subsection{Cultural Requirements}
N.A.
\subsection{Legal Requirements}
\begin{enumerate}[{LR}1. ]
    \item The system must not keep data if a user retracts consent. \\
    \textbf{Fit Criteria}:  Upon request to delete data, there should be no instance of the user's data in the database.
\end{enumerate}

\subsection{Health and Safety Requirements}
N.A.\\

\subsection{Requirements That Are Likely/Unlikely to Change}
\begin{table}[h]
    \centering
    \begin{tabular}{|p{0.40\linewidth} | p{0.40\linewidth}|}
    \hline
         Likely & Unlikely \\
         \hline
         PR1, PR2, SR1 & LF1, LF2, LF3, LF4, LF5, LF6, UH1, OE1, OE2, MR1, LR1      \\
         \hline
    \end{tabular}
    \caption{Likely/Unlikely to Change Table}
    \label{tab:my_label}
\end{table}
\begin{itemize}
    \item LF1, LF2, LF3, LF4, LF6: These are unlikely to change as these are needed to fully immerse the user within the game and not think about the underlying research (to ensure genuine results).
    \item LF5, UH1, OE1, OE2: This is unlikely to change as these tackle the problem of having the application as accessible as possible.
    \item MR1: This is unlikely to change as it allows Dr.Yiannakoulias to tweak and further improve upon the finished product.
    \item LR1: This is unlikely to change as this is required for the ethics board to approve the finished product.
    
\end{itemize}

\section{Traceability Matrix}

The traceability matrix below shows the relationships between functional requirements and non-functional requirements/risks. Any cells with a 'Y' trace to each other. 

\begin{table}[H]
\begin{adjustwidth}{-1.87in}{-1in}
\begin{tabular}{c|c|c|c|c|c|c|c|c|c|c|c|c|c|c|c|c|c|c|}
\cline{2-19}
                                    & \textbf{LF1} & \textbf{LF2} & \textbf{LF3} & \textbf{LF4} & \textbf{LF5} & \textbf{LF6} & \textbf{UH1} & \textbf{PR1} & \textbf{PR2} & \textbf{OE1} & \textbf{OE2} & \textbf{MS1} & \textbf{SR1} & \textbf{LR1} & \textbf{R1} & \textbf{R2} & \textbf{R3} & \textbf{R4} \\ \hline
\multicolumn{1}{|c|}{\textbf{FR1}}  &              &              &              &              &              &              &              &              &              &              &              &              & Y            &              &             &             &             &             \\ \hline
\multicolumn{1}{|c|}{\textbf{FR2}}  &              &              &              &              &              &              &              &              &              &              &              &              &              &              &             &             &             &             \\ \hline
\multicolumn{1}{|c|}{\textbf{FR3}}  &              &              &              &              &              & Y            &              &              &              &              &              &              &              &              &             &             &             &             \\ \hline
\multicolumn{1}{|c|}{\textbf{FR4}}  &              &              &              &              &              &              &              &              &              &              &              &              &              &              &             &             &             &             \\ \hline
\multicolumn{1}{|c|}{\textbf{FR5}}  &              &              &              &              &              &              &              &              &              &              &              &              &              &              &             &             &             &             \\ \hline
\multicolumn{1}{|c|}{\textbf{FR6}}  &              &              &              &              &              & Y            &              &              &              &              &              &              &              &              &             &             &             &             \\ \hline
\multicolumn{1}{|c|}{\textbf{FR7}}  &              &              &              &              &              & Y            &              &              &              &              &              &              &              &              &             &             &             &             \\ \hline
\multicolumn{1}{|c|}{\textbf{FR8}}  &              &              &              &              &              & Y            &              &              &              &              &              &              &              &              &             &             &             &             \\ \hline
\multicolumn{1}{|c|}{\textbf{FR9}}  &              &              &              &              &              & Y            &              &              &              &              &              &              &              &              &             &             &             &             \\ \hline
\multicolumn{1}{|c|}{\textbf{FR10}} &              &              &              &              &              & Y            &              &              &              &              &              &              &              &              &             &             &             &             \\ \hline
\multicolumn{1}{|c|}{\textbf{FR11}} &              &              &              &              &              &              &              &              &              &              &              &              &              &              &             &             &             &             \\ \hline
\multicolumn{1}{|c|}{\textbf{FR12}} &              &              &              &              &              &              &              &              &              &              &              &              &              &              &             &             &             &             \\ \hline
\multicolumn{1}{|c|}{\textbf{FR13}} &              &              &              &              &              &              &              &              &              &              &              &              &              &              &             &             &             &             \\ \hline
\multicolumn{1}{|c|}{\textbf{FR14}} &              &              &              &              &              &              &              &              &              &              &              &              &              &              &             &             &             &             \\ \hline
\multicolumn{1}{|c|}{\textbf{FR15}} &              &              &              &              &              &              &              &              &              &              &              &              &              & Y            &             & Y           &             &             \\ \hline
\multicolumn{1}{|c|}{\textbf{FR16}} &              &              &              &              &              & Y            &              &              &              &              &              &              &              &              &             &             &             &             \\ \hline
\multicolumn{1}{|c|}{\textbf{FR17}} &              &              &              &              &              & Y            &              &              &              &              &              &              &              &              &             &             &             &             \\ \hline
\multicolumn{1}{|c|}{\textbf{FR18}} &              &              &              &              &              &              &              &              &              &              &              &              &              &              &             & Y           &             &             \\ \hline
\multicolumn{1}{|c|}{\textbf{FR19}} &              &              &              &              &              &              &              &              &              &              &              &              &              &              &             &             &             &             \\ \hline
\multicolumn{1}{|c|}{\textbf{FR20}} &              &              &              &              &              &              &              &              &              &              &              &              &              &              &             &             &             &             \\ \hline
\multicolumn{1}{|c|}{\textbf{FR21}} &              &              &              &              &              & Y            &              &              &              &              &              &              &              &              &             &             &             &             \\ \hline
\multicolumn{1}{|c|}{\textbf{FR22}} &              &              &              &              &              &              &              &              &              &              &              &              &              &              &             &             &             &             \\ \hline
\multicolumn{1}{|c|}{\textbf{R1}}   &              &              &              &              &              &              &              &              &              &              &              &              &              &              &             &             &             &             \\ \hline
\multicolumn{1}{|c|}{\textbf{R2}}   &              &              &              &              &              &              &              &              &              &              &              &              &              &              &             &             &             &             \\ \hline
\multicolumn{1}{|c|}{\textbf{R3}}   &              &              & Y            & Y            &              & Y            &              &              &              &              &              &              &              &              &             &             &             &             \\ \hline
\multicolumn{1}{|c|}{\textbf{R4}}   &              &              &              &              &              &              &              &              &              &              &              &              &              &              &             &             &             &             \\ \hline
\end{tabular}
\caption{Traceability Matrix}
    \label{tab:my_label}
\end{adjustwidth}
\end{table}

\section{Project Issues}

\subsection{Open Issues}

\subsection{Off-the-Shelf Solutions}

\subsection{New Problems}

\subsection{Tasks}

\subsection{Migration to the New Product}

\subsection{Risks}

\subsection{Costs}

\subsection{User Documentation and Training}

\subsection{Waiting Room}

\subsection{Ideas for Solutions}

\bibliographystyle{plainnat}

\bibliography{SRS}

\newpage

\section{Appendix}

\subsection{Team technologies/skills - Reflection  }
\begin{enumerate}
    \item \textbf{ReactJS}: To master this technology, the following members will go through an in-depth tutorial on ReactJS \href{https://reactjs.org/tutorial/tutorial.html}{Tutorial: Intro to React} and complete a basic project that can be found on YouTube (i.e Create a basic sign-up/login-page) to exercise different components taught in the ReactJS tutorial.
        \begin{itemize}
            \item Student: Mihail
        \end{itemize}
    
    \item \textbf{Team Management}: To master this skill, the following members will go through an online LinkedIn course about team management and effective team leadership \href{https://www.linkedin.com/learning/topics/leadership-and-management}{Leadership and Management Online Training Courses}. Another approach is to take on a leadership role when completing the mini projects above that will be done in small groups of 3 or 4 for each group. This skill was chosen because team management is a large factor in a project's success that is often overlooked.
        \begin{itemize}
                \item Student: Brandon
            \end{itemize}
    \item \textbf{Web Development}: To master this technology, the following members will go through a series of basic web development fundamentals and concepts such as \href{https://www.w3schools.com/html/default.asp}{HTML Tutorial}, \href{https://www.w3schools.com/css/default.asp}{CSS Tutorial} and \href{https://www.w3schools.com/js/default.asp}{JavaScript Tutorial} provided by W3Schools. As well, a mini project suggested above will be used to demonstrate the understanding and skills learned through the tutorials. This skill was chosen because this member had very little experience with this tech stack and learning this could be helpful for further projects and work.
        \begin{itemize}
                \item Student: Mohammad 
            \end{itemize}
    \item \textbf{ExpressJS}: To master this technology, the following members will go through an in-depth tutorial and documentation \href{https://expressjs.com/}{Express - Node.js Web Application Framework}. As well, a mini project suggested above will be used to demonstrate the understanding and skills learned through the in-depth tutorial and documentation. This technology was chosen because the following members had a small experience with Express.js and would like to further get a better understanding of middleware in web development, routing, and ultimately better at server-side development.
        \begin{itemize}
                \item Student: Andrew 
            \end{itemize}
    \item \textbf{SQL Database}: To master this technology, the following members will go through an in-depth tutorial for \href{https://www.w3schools.com/sql/default.asp}{SQL}. They will also get hands-on experience by working on a mini-project that will leverage SQL to demonstrate the understanding and skills learned through the tutorial. If needed, other team members with prior SQL experience can assist. This technology was chosen as it is relevant to the industry and will be used in the project.
        \begin{itemize}
                \item Student: Namit
            \end{itemize}
\end{enumerate}


\subsection{Symbolic Parameters}

The definition of the requirements will likely call for SYMBOLIC\_CONSTANTS.
Their values are defined in this section for easy maintenance.


\begin{table}[h]
\caption{\bf Symbolic Parameter Table}
\begin{tabular}{|l|p{0.5\linewidth}|l|}
\hline
\multicolumn{1}{|l}{\bfseries Symbolic Parameter} & \multicolumn{1}{|l|}{\bfseries Description} & \multicolumn{1}{l|}{\bfseries Value}\\
\hline
AGE\_GROUP & The age of the users playing the game & 18 and above \\
\hline
MIN\_MONITOR & The minimum supported monitor resolution for responsiveness & 1280 by 720 \\
\hline
INACTIVE\_TIME & The time used to establish an inactive user. No data has been logged or any control inputs (mouse or keyboard) by the user & 15 minutes \\
\hline
LOAD\_TIME & The maximum time allowed for the application to successfully load & 5 seconds \\
\hline
RESPONSE\_TIME & The maximum time allowed for the application to respond to user input & 5 seconds \\
\hline
CONSULTING\_INTERVAL & The number of turns between each consultant visit & 3 turns \\
\hline
EVENT\_OCCURRENCE & The amount of times an event will occur throughout a season & 2/season \\
\hline
SEASON\_LENGTH & The number of turns per season & 3 turns \\
\hline
SUPPORTED\_VERSIONS & The oldest supported version of browsers & 1 year \\
\hline
MIN\_TURNS & The minimum amount of turns played needed for a study participant to be a significant data point & 12 turns \\
\hline
\end{tabular}
\end{table}


\end{document}
